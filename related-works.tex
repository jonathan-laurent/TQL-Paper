\section{Related Works}

Languages already exist to collect, filter and aggregate data from
simulation. For example, the Kappa simulator \cite{BoutillierEK17}
features an \emph{intervention language} that -- among other things --
offers primitives to take repeated conditional measurements during
simulation, possibly updating a global state every time. The Chromar
\cite{honorato2017chromar} rule-based language proposes an
\emph{extended language} with similar capabilities, where queries are
written in a more functional style that is reminiscent of
MapReduce. Finally, languages have been proposed
\cite{helms2012toward,zehe2016online} to specify the same task in a
fully declarative way, that are inspired by the {structured query
  language} (\textsc{sql}).

All these languages are similar in the sense that
\begin{inparaenum}[(i)]
\item queries select individual states or transitions and perform
  an action on each of them
\item only population-level quantities can be measured.
\end{inparaenum}
In contrast, queries in our proposed language
%\begin{inparaenum}[(i)]
can select \emph{trace patterns} that consist in multiple related
transitions acting on common agents and allow measuring and comparing
the state of individual agents at different points in time. Therefore,
most of the example queries shown in this paper could not be expressed
using preexisting query languages.

Thanks to its ability to match complex trace patterns, our query
language may be especially interesting to use in combination with
causal analysis
\cite{DanosEtAl-CONCUR07,DBLP:conf/fsttcs/DanosFFHH12}. Indeed, the
pathways uncovered by causal analysis can be regarded as trace
patterns and then matched using our query language. For example, query
(\ref{query:cat-deg}) in section~\ref{sec:use-case} compares the
composition of the complex at which $\beta$-catenin is attached at
different points of the pathway leading to its degradation.

