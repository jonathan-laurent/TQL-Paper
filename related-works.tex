\section{Related Works}

Languages already exist to collect, filter and aggregate data from
simulation. For example, the Kappa simulator \cite{BoutillierEK17}
features an \emph{intervention language} that allows taking repeated
conditional measurements during simulation, possibly updating a global
state every time. Chromar \cite{honorato2017chromar} proposes an
\emph{extended language} with similar capabilities, where queries can
be defined in a more functional style from a selector and a fold
operation. Finally, query languages have been proposed
\cite{helms2012toward,zehe2016online} that are inspired by the
{structured query language} (\textsc{SQL}).

All these languages are similar in the sense that
\begin{inparaenum}[(i)]
\item queries can only evaluate expressions in the scope of a single
state or transition and
\item only population-level quantities can be measured.
\end{inparaenum}
In contrast, queries in our proposed language can
\begin{inparaenum}[(i)]
\item match \emph{trace patterns} that consist in multiple transitions
  acting on common agents and
\item measure and compare the state of individual agents, possibly at
  different points in time.
\end{inparaenum}
As a consequence, most of the example queries shown in this paper
could not be expressed using preexisting query languages.

Thanks to its ability to match complex trace patterns, our query
language may be especially interesting to use in combination with
causal analysis
\cite{DanosEtAl-CONCUR07,DBLP:conf/fsttcs/DanosFFHH12}. Indeed, the
pathways uncovered by causal analysis can be regarded as trace
patterns and then matched using our query language. This may be useful
to measure the relative frequency of pathways in different settings,
but also to analyze how individual agents participating to a pathway
evolve as this pathway unfolds. For example, query
(\ref{query:cat-deg}) in section~\ref{sec:use-case} compares the
composition of the complex at which $\beta$-catenin is attached at
different points of the pathway leading to its degradation.
