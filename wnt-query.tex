%%%%%%%%%%%%%%%%%%%%%%%%%%%%%%%%%%%%%%%%%%%%%%%%%%%%%%%%%%%%%%%%%%%%%%%%%%%%%%%%

\subsubsection*{Wait times}

What is the distribution of times spent
between the first phosphorylation on an agent, and the time it gets
degraded?

%match i:{+c:Cat}
%and first p:{c:Cat(S45{un/ph})} after i
%and first d:{-c:Cat} after p
%return (agent_id{c}, time[p], time[d])

\begin{small}
\begin{equation}
  \Query{
    \m{match} & i:\Set{  c:\BetaCat+ } \\
    \m{and} & \FirstAfter{p:\Set{
        \iAG{c}{\BetaCat}{{S45}_{\Trans{u}{p}}}
    }}{i} \\
    \m{and} & \FirstAfter{ d:\Set{
        c:\BetaCat-
    } }{p} \\
    \m{return} & \Par{\AgentId{c},\, \Time{p},\, \Time{d}}
  } 
\end{equation}
\end{small}

\noindent \textit{About this query.} Agent creation and destruction is
expressed by prefixing agent names with $+$ and $-$,
respectively. Moreover, \textsf{agent\_id} is a function that returns
the unique integer identifier of an agent.

%%%%%%%%%%%%%%%%%%%%%%%%%%%%%%%%%%%%%%%%%%%%%%%%%%%%%%%%%%%%%%%%%%%%%%%%%%%%%%%%

\subsubsection*{Undoing S45, T41, S37 and S33 phosphorylation}
Considering phosphatases are around and undoing the phosphorylation of
sites, does this happen to all agents? Does it happen to just an
unlucky few number of agents? What is the distribution of
dephosphorylation events per agent?

%match e:{c:Cat(S45{ph/un})}
%return (agent_id{c}, time[e])

%match e:{c:Cat(T41{ph/un})}
%return (agent_id{c}, time[e])

%match e:{c:Cat(S37{ph/un})}
%return (agent_id{c}, time[e])

%match e:{c:Cat(S33{ph/un})}
%return (agent_id{c}, time[e])

\newcommand{\UndoQ}[1]{
\Query{
    \m{match} & e:\Set{ 
      \iAG{c}{\BetaCat}{{#1}^{\,1}_{\Trans{p}{u}}}
    } \\
    \m{return} & \Par{\AgentId{c},\, \Time{e}}
  }
}

\begin{small}
  \begin{equation*}
    \arraycolsep=7pt
    \begin{array}{cc}
      \UndoQ{S45} & \UndoQ{T41} \\ \\
      \UndoQ{S37} & \UndoQ{S33} \\
    \end{array}
  \end{equation*}
\end{small}


%%%%%%%%%%%%%%%%%%%%%%%%%%%%%%%%%%%%%%%%%%%%%%%%%%%%%%%%%%%%%%%%%%%%%%%%%%%%%%%%

\subsubsection*{Component size and enzyme identity} 
Where do the phosphorylation steps that actually lead to degradation
occur? Do they happen overwhelmingly on the large complexes? What’s
the composition in units of Axn and APC of the complexes where the
phosphorylation events leading to degradation took place?  Which
enzymes contribute most? Are all enzymes contributing comparably to
the destruction, or are some continuously sabotaged by phosphatases?
What is the distribution of kinase identifiers for the last
phosphorylation events that lead to degradation?

\newcommand{\BigHectorStoryLine}[4]{
\LastBefore{#1:\Set{ 
          \iAG{c}{\BetaCat}{ {#2}^{\,1}_{\Trans{u}{p}}}, \ 
          \iAG{#3}{#4}{c^{\,1}}
    }}{d}
}
\newcommand{\BigHectorStoryRet}[2]{
\AgentId{#1}, \ \Count{ \Component{\StateBefore{#2}}{#1}, \, 
      \STR{Axn}, \, \STR{APC} }
}

%match d:{-c:Cat}
%and last p1:{ c:Cat(S45{un/ph}[1]), k1:CK1(c[1])} before d
%and last p2:{ c:Cat(T41{un/ph}[1]), k2:GSK(c[1])} before d
%and last p3:{ c:Cat(S37{un/ph}[1]), k3:GSK(c[1])} before d
%and last p4:{ c:Cat(S33{un/ph}[1]), k4:GSK(c[1])} before d
%return (
%	agent_id{k1}, count{'Axn', 'APC'}{component[.p1]{k1}},
%	agent_id{k2}, count{'Axn', 'APC'}{component[.p2]{k2}},
%	agent_id{k3}, count{'Axn', 'APC'}{component[.p3]{k3}},
%	agent_id{k4}, count{'Axn', 'APC'}{component[.p4]{k4}}

\begin{small}
\begin{equation}
  \Query{
    \m{match} & d:\Set{ c:\BetaCat - } \\
    \m{and} & \BigHectorStoryLine{p_1}{S45}{k_1}{\CKOne} \\
    \m{and} & \BigHectorStoryLine{p_2}{T41}{k_2}{\GSK}   \\
    \m{and} & \BigHectorStoryLine{p_3}{S37}{k_3}{\GSK}   \\
    \m{and} & \BigHectorStoryLine{p_4}{S33}{k_4}{\GSK}   \\
    \m{return} 
    & \BigHectorStoryRet{p_1}{k_1} \,,\  \\
    & \BigHectorStoryRet{p_2}{k_2} \,,\ \\
    & \BigHectorStoryRet{p_3}{k_3} \,,\  \\
    & \BigHectorStoryRet{p_4}{k_4} \\
  }
\end{equation}
\end{small}

\noindent \textit{About this query.} The \textsf{component} state
measure computes the connected component that contains an agent in a
mixture. It returns a set of agents $S$. The \textsf{count} function
takes such a set $S$ along with $n$ strings denoting agent types and
returns an $n$-tuple of integers indicating how many agents of each
type appear in $S$.
