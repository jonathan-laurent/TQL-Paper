\section{Conclusion}

How could one have explored a question like ``which complexes
contribute to kinetics''? Experimental biologists have been using
labeling techniques for decades, but implementing this in a modeling
framework requires being able to track individual agents, and query
particular events. Implementing a framework to query events on the
trace of an agent and rule simulation seems a natural way of tackling
these classes of problems. Moreover, once a sufficiently rich
\emph{mechanistic} model is available, questions on \emph{mechanism}
arise. For a subset of these, a satisfying answer will require a
change of vocabulary; the explanations desired use the individual's
lexicon (e.g. it bound, it unbound, it got dephosphorylated 800
times), rather than a whole system lexicon (e.g. the abundance changed
from 500 to 50). Thus, rather than tracking the whole model's behavior
(akin to a top-down approach), one needs to focus on agents, and
observe their individual experiences (akin to a bottom-up
approach). These approaches are complementary, as they explore a
model's intricacies from very different viewpoints. We hope that the
query language proposed in this paper will contribute to make
agent-centric analyses more widespread and accessible.
