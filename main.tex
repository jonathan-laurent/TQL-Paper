\documentclass[runningheads]{llncs}
\usepackage{graphicx}
\usepackage{hyperref}
\usepackage{stmaryrd}
\usepackage{amsmath}
\usepackage{esvect}
\usepackage{listings}
\usepackage{paralist}
\usepackage{multicol}
\usepackage{subcaption}

% -*- TeX-master: "main.tex" -*-

%%%%%%%%%%%%%%%%%%%%%%%%%%%%%%%%%%%%%%%%%%%%%%%%%%%%%%%%%%%%%%%%%%%%%%%%%%%
% Trace Query Language

\newcommand{\MatchRet}[2]{\textsf{match} \ #1 \ \textsf{return} \  #2}
\newcommand{\FirstAfter}[2]{\textsf{first} \,\ #1 \ \textsf{after} \ #2}
\newcommand{\LastBefore}[2]{\textsf{last} \,\ #1 \ \textsf{before} \ #2}

\newcommand{\Time}[1]{\textsf{time}\SqPar{#1}}
\newcommand{\Rule}[1]{\textsf{rule}\SqPar{#1}}
\newcommand{\IntState}[3]{\textsf{int\_state}\SqPar{#1}\Par{#2, #3}}
\newcommand{\Component}[2]{\textsf{component}\SqPar{#1}\Par{#2}}
\newcommand{\Snapshot}[1]{\textsf{snapshot}\SqPar{#1}}
\newcommand{\AgentId}[1]{\textsf{agent\_id}\Par{#1}}
\newcommand{\Count}[1]{\textsf{count}\Par{#1}}


\newcommand{\TracePat}{P}
\newcommand{\TransPat}{T}

%%%%%%%%%%%%%%%%%%%%%%%%%%%%%%%%%%%%%%%%%%%%%%%%%%%%%%%%%%%%%%%%%%%%%%%%%%%
% Semantics

\newcommand{\Event}[0]{\textsf{Event}}
\newcommand{\Transition}[0]{\textsf{Transition}}
\newcommand{\Matching}[0]{\textsf{Matching}}
\newcommand{\Bool}[0]{\textsf{Bool}}
\newcommand{\Trace}[0]{\textsf{Trace}}
\newcommand{\Value}[0]{\textsf{Value}}
%\newcommand{\Vector}[1]{\vv{#1}}
\newcommand{\Vector}[1]{\vec{#1}}
\newcommand{\At}[2]{{#1}{[#2]}}

\newcommand{\Ag}[0]{\,\textsf{a}}
\newcommand{\Ev}[0]{\,\textsf{E}}
\newcommand{\Tr}[0]{\,\textsf{t}}


\newcommand{\Inter}{\,\cap\,}


\newcommand{\Query}[1]{
\ensuremath{
\arraycolsep=5pt
\def\arraystretch{1.4}
\begin{array}{ll}
#1
}
\end{array}}

%%%%%%%%%%%%%%%%%%%%%%%%%%%%%%%%%%%%%%%%%%%%%%%%%%%%%%%%%%%%%%%%%%%%%%%%%%%
% Maths

\newcommand{\Par}[1]{\left(\, #1  \,\right)}
\newcommand{\SqPar}[1]{\left[\, #1  \,\right]}
\newcommand{\Set}[1]{\left\{\, #1  \,\right\}}
\newcommand{\SetC}[2]{\left\{\, #1 \ | \ #2  \,\right\}}
\newcommand{\Sem}[1]{\left\llbracket \, #1 \, \right\rrbracket}
\newcommand{\Bag}[1]{\Lbag \, #1 \, \Rbag}
\newcommand{\BagC}[2]{\Lbag \, #1 \ | \ #2  \,\Rbag}
\newcommand{\m}[1]{\textsf{#1}}
\newcommand{\OR}[0]{\ \ | \ \ }
\newcommand{\STR}[1]{\texttt{"#1"}}

\newcommand{\MultilineSetC}[3]{
  \ensuremath{
  \def\arraystretch{1.5}
  \begin{array}{ll}
    \{\, #1 \ | \ & #2 \\
    & #3 \,\}
  \end{array}}
}

\newcommand{\LargeMultilineSetC}[3]{
  \ensuremath{
  \def\arraystretch{1.35}
  \Bigg\{
  \, \  #1 \ \ \Big| \ \
  \begin{array}{l}
    #2 \\
    #3
  \end{array}
  \ 
  \Bigg\}
  }
}

\newcommand{\AG}[2]{{#1}\Par{#2}}
\newcommand{\iAG}[3]{#1\!:\!{#2}\Par{#3}}

\newcommand\sbullet[1][.5]{\mathbin{\vcenter{\hbox{\scalebox{#1}{$\bullet$}}}}}
\newcommand{\Free}{\sbullet[.6]}
\newcommand{\Trans}[2]{\, #1 \, \to \, #2 }
\newcommand{\SlashTrans}[2]{\, #1 \, / \, #2 \,}

\newcommand{\StateBefore}[1]{\sbullet[.6]\,{#1}}
\newcommand{\StateAfter}[1]{{#1}\,\sbullet[.6]}



\usepackage{array}

% https://tex.stackexchange.com/questions/12703/how-to-create-fixed-width-table-columns-with-text-raggedright-centered-raggedlef
\newcolumntype{L}[1]{>{\raggedright\let\newline\\\arraybackslash\hspace{0pt}}m{#1}}
\newcolumntype{C}[1]{>{\centering\let\newline\\\arraybackslash\hspace{0pt}}m{#1}}
\newcolumntype{R}[1]{>{\raggedleft\let\newline\\\arraybackslash\hspace{0pt}}m{#1}}


%\newcommand{\BetaCat}{$\textit{Cat}$}
\newcommand{\BetaCat}{\text{Cat}}
\newcommand{\CKOne}{\text{CK1}}
\newcommand{\GSK}{\text{GSK}}




\renewcommand\UrlFont{\color{blue}\rmfamily}

\begin{document}

\title{A Trace Query Language for Kappa}

%\author{Jonathan Laurent\inst{1} \and Jean Yang\inst{1,2}}
\author{Anonymous Authors}

%\authorrunning{Jonathan Laurent et al.}
\authorrunning{Anonymous Authors}


%\institute{Carnegie Mellon University \and Harvard Medical School}\\

\maketitle

%%%%%%%%%%%%%%%%%%%%%%%%%%%%%%%%%%%%%%%%%%%%%%%%%%%%%%%%%%%%%%%%%%%%%%%%%%%

%TODO: novel/unified
\begin{abstract}
  In this paper, we introduce a unified approach to querying
  simulation traces of rule-based models, along with an implementation
  for the Kappa language. In our approach, a query consists in a trace
  pattern along with an expression that depends on the variables
  captured by this pattern. On a given trace, it evaluates to the
  multiset of all values of the expression for every possible matching
  of the pattern. We illustrate our query language on a simple
  example, and then discuss its semantics and implementation. Finally,
  we provide a detailed use case where we analyze the dynamics of
  $\beta$-catenin degradation in Wnt signaling.

  \keywords{Kappa \and Rule-based modelling \and Query Language.}
\end{abstract}

%%%%%%%%%%%%%%%%%%%%%%%%%%%%%%%%%%%%%%%%%%%%%%%%%%%%%%%%%%%%%%%%%%%%%%%%%%%

\section{Introduction}

Rule-based modeling languages such as Kappa \cite{DanosEtAl-CONCUR07}
and BioNetGen \cite{bngl} can be used to write mechanistic models of
complex reaction systems. Models in these languages consist in
stochastic graph-rewriting rules that are equipped with rate constants
indicating their propensity to apply. Together with an initial mixture
graph, these rules constitute a dynamical system that can be simulated
using Gillespie's algorithm
\cite{gillespie1977exact,DanosEtAl-APLAS07,BoutillierEK17}. Each run
of simulation results in a sequence of transitions that we call a
trace.

In practice, simulation traces are often discarded in favor of a
limited number of features such as the concentration curves of a set
of observables. However, a more detailed analysis of their structure
and statistical properties can provide useful insights into a system's
dynamics. For example, causal analysis methods exist
\cite{DanosEtAl-CONCUR07,DBLP:conf/fsttcs/DanosFFHH12} that compress a
large trace into a minimal subset of events that are necessary and
jointly sufficient to replicate an ouctome of interest, and then
highlight causal influences between those remaining events.
% Ideally, such techniques provide a way to uncover signalling
% pathways from networks of low-level protein-protein interactions
Queries about the statistical behavior of individual agents can also
lead to complementary insights. Examples include
\begin{inparaenum}[(i)]
\item measuring the average lifespan of a complex under different
  conditions,
\item computing a probability distribution over the states in which a
  particular type of agent can be when targeted by a given rule, and
\item estimating how much of a certain kind of substrate getting
  phosphorylated is due to a particular pathway at different points in
  time.
\end{inparaenum}

% Engineering challenge.
% Flexible tool.
% Extensible.
% Principled foundations -> can evolve in the future.


\section{A Starting Example}\label{sec:starting-example}

In order to illustrate our Trace Query Language, we introduce a toy
Kappa model in Figure~\ref{fig:model}. It is described using a rule
notation that has been introduced in the latest release of the Kappa
simulator and which we borrow in our query language. With this
notation, a rule is described as a pattern that is annotated with
rewriting instructions. The pattern denotes a precondition that is
required for a rule to target a collection of agents. Rewriting
instructions are specified by arrows that indicate the new state of a
site after transformation.

The model of Figure~\ref{fig:model} features two types of agents:
substrates $S$ and kinases $K$. Both kinds of agents have two
different sites, named $x$ and $d$. In addition, $x$-sites can be in
two different internal states: \textit{unphosphorylated} and
\textit{phosphorylated}. We write those states $u$ and $p$,
respectively. Rule $b$ expresses the fact that a substrate and a
kinase with free $d$-sites can bind at rate $\lambda_b$. Rules $u$ and
$u^*$ express the fact that the breaking of the resulting complex
happens at different rates, depending on the phosphorylation state of
the kinase involved. Finally, rule $p$ expresses the fact that a
substrate that is bound to a kinase can get phosphorylated at rate
$\lambda_p$. In all our examples, we consider initial mixtures
featuring free substrates and kinases in smiliar quantity. Substrates
are initially unphosphorylated and kinases are present in both
phosphorylation states.


% -*- TeX-master: "main.tex" -*-
\begin{figure}
  \vspace{-0.2cm}
  \begin{subfigure}[h]{0.45\linewidth}
    \def\arraystretch{1.8}
    \begin{small}
      \begin{tabular}{lllcl}
        \multicolumn{5}{c}{$\lambda_u \gg \lambda_{u^*} \approx \lambda_p$} \\
        $b$ \hspace{0.05cm} & : \hspace{0.1cm} & $ \AG{S}{d^{\Trans{\Free}{1}}}, \ 
        \AG{K}{d^{\Trans{\Free}{1}}} $ &  \ @ \  & $\lambda_b$ \\
        $u$ & : & $ \AG{S}{d^{\Trans{1}{\Free}}}, \ 
                  \AG{K}{d^{\Trans{1}{\Free}}, \ x_u} $ & @ & $\lambda_u$ \\
        $u^*$ & : & $ \AG{S}{d^{\Trans{1}{\Free}}}, \ 
                    \AG{K}{d^{\Trans{1}{\Free}}, \ x_p} $ & @ & $\lambda_{u^*}$ \\
        $p$ & : & $ \AG{S}{d^{\,1}, \  x_{\Trans{u}{p}}}, \ 
                  \AG{K}{d^{\,1}} $ & @ & $\lambda_p$ \\
      \end{tabular}
    \end{small}
    % \subcaption{In Prettified Kappa}
  \end{subfigure}
  \hfill
  \begin{subfigure}[h]{0.43\linewidth}
    \begin{center}
      \includegraphics[scale=0.8]{kappa-diagrams/model.pdf}
    \end{center}
    % \subcaption{Graphical Representation}
  \end{subfigure}
  \caption{An Example Kappa Model. On the left, it is described using
    the \textit{edit notation} introduced in KaSim 4. Numbers in a
    rule expression correspond to local bond identifiers and $\Free$
    indicates a free site. Sites not mentioned in a rule are left
    unchanged by it. A graphical representation is provided on the
    right. Phosphorylated sites are indicated in grey. Dotted and
    solid arrows indicate \textit{slow} and \textit{fast} reactions,
    respectively.  }\label{fig:model}
\end{figure}

By playing with this model a bit, one may notice that the
concentration of phosphorylated substrate reaches its maximal value
faster when the ratio of phosphorylated kinases is high (given the
rules of our model, the latter quantity is invariant during
simulation). This phenomenon cannot be explained by looking at rule
$p$ alone. The query provided in~(\ref{eq:exq1}) can be run to
estimate the probability that a substrate is bound to a phosphorylated
kinase when it gets phosphorylated:
\begin{equation}\label{eq:exq1}
  \m{match } t:\Set{ \AG{S}{x_{\Trans{u}{p}},\ d^{\,1}},\ \iAG{k}{K}{d^{\,1}} }
  \, \m{ return } \, \IntState{\StateBefore{t}}{k}{\STR{x}}
\end{equation}
Given a trace, this query matches every transition where a substrate
is getting phosphorylated and outputs the phosphorylation state of the
attached kinase. The variables $t$ and $k$ denote a transition and an
agent, respectively. Moreover, the expression
$\IntState{\,\StateBefore{t}\,}{\,k\,}{\STR{x}\,}$ refers to the
internal state of the site of agent $k$ with name \STR{x} in the
mixture preceding transition $t$.

Running the previous query, we learn that substrates are much more
likely to be phosphorylated by kinases that are phosphorylated
themselves, even when such kinases are in minority in the mixture.
This leads us to conjecture a causal link between the phosphorylation
state of a kinase and its efficiency. After some thoughts, this link
can be easily interpreted: because $\lambda_u \gg \lambda_{u^*}$,
phosphorylated kinases form more stable complexes with substrates,
leaving more chances for a phosphorylation interaction to happen.  In
fact, the average lifespan of a kinase-substrate complex is exactly
$\lambda_{u^*}^{-1}$ when the kinase is phosphorylated and
$\lambda_u^{-1}$ when it is not. We can check these numbers
experimentally by running the following query:
\begin{equation}\label{eq:query-lifespan}
  \Query{
    \m{match} & 
    b:\Set{ \iAG{s}{S}{d^{\Trans{\Free}{1}}}, \ 
      \AG{K}{d^{\Trans{\Free}{1}}, \ x_{p}} } \\
    \m{and} &
    \FirstAfter{u:\Set{ \iAG{s}{S}{d^{\Trans{}{\Free}}} }}{b} \\
    \m{return} & 
    \Time{u} - \Time{b}
  }
\end{equation}
This query outputs a multiset of numbers, whose mean is the average
lifespan of a complex formed by a substrate and a phosphorylated
kinase. The same quantity can be computed for unphosphorylated kinases
by replacing $x_p$ by $x_u$ in the first
line. %of~\ref{eq:query-lifespan}.
The pattern in this query does not match single transitions but pairs
of related transitions $(b, u)$, where $b$ is a binding transition and
$u$ the first unbinding transition to target the same substrate.

\iffalse
% Additional example
\begin{equation}\label{eq:query-kinase-efficiency}
  \Query{
    \m{match} & 
    b:\Set{ \iAG{s}{S}{x_u,\ d^{\Trans{\Free}{1}}}, \ 
      \iAG{k}{K}{d^{\Trans{\Free}{1}}} } \\
    \m{and} &
    \FirstAfter{u:\Set{ \iAG{s}{S}{d^{\Trans{}{\Free}}} }}{b} \\
    \m{return} & 
      \IntState{\StateBefore{b}}{k}{\STR{x}}, \ 
      \IntState{\StateAfter{u}}{s}{\STR{x}}
  }
\end{equation}
\fi

\bigskip

More generally, a query is defined by a {pattern}
$P[\Vec{t}, \Vec{a}]$ and an {expression} $E[\Vec{t}, \Vec{a}]$, which
feature a shared set $\Vec{t}$ of transition variables and a shared
set $\Vec{a}$ of agent variables. The pattern $P$ can be regarded as a
predicate that takes as its arguments a trace $\tau$ and a
\emph{matching} $\phi$ mapping the variables in $\Vec{t}$ and
$\Vec{a}$ to actual transitions and events in $\tau$.  The expression
$E$ can be regarded as a function that maps such $(\tau, \phi)$ pairs
to values. Then, the query evaluates on a trace $\tau$ to the multiset
of all values of $E$, for every matching $\phi$ that satisfies $P$ in
$\tau$.

% TODO: values are tuples, generate a CSV file

\section{The Core Query Language}\label{sec:semantics}

In this section, we introduce the extensible core of our proposed
query language and give it a formal semantics.

\subsection{Meaning and Structure of Queries}\label{subsec:structure}
As shown in Figure~\ref{fig:semantics}, a query $Q$ consists in a
pattern $P$ and an expression $E$. It can be interpreted as a function
$\Sem{Q}$ from traces to multisets\footnote{Note that multisets are
  indicated in Figure~\ref{fig:semantics} using Dijkstra's bag
  notation, whereas sets are indicated using the standard curly
  brackets notation.} of values. The set of allowed values can grow
larger as richer expressions are added to the language. Our current
implementation defines a value as a tuple of base values and features
the following types for base values: \m{bool}, \m{int}, \m{float},
\m{string}, \m{agent}, \m{agent\_set} and \m{snapshot}.

A pattern $P$ is interpreted as a function $\Sem{P}$ that maps a trace
to a set of matchings. A matching $\phi$ is defined by two functions
$\phi_{\Tr}$ and $\phi_{\Ag}$, which map variable names to transition
identifiers and agent identifiers, respectively. We call $\phi_{\Tr}$
a transition matching and $\phi_{\Ag}$ an agent matching.  Given a
trace $\tau$ and a matching $\phi$, the transition variable $v$
denotes the transition $\tau[\phi_{\Tr}(v)]$, where $\tau[i]$ is a
notation for the $i^{th}$ transition of a trace. In addition, an
expression $E$ is interpreted as a function $\Sem{E}$ that maps a pair
of a trace and a matching to a value. The expression language is
extensible and is discussed in section~\ref{subsec:expr-language}. Its
syntax is documented in Figure~\ref{fig:expressions}. Then, the
semantics of a query can be formally defined as follows:
\[ \Sem{\MatchRet{\TracePat}{E}}(\tau) \ = \ \BagC{\Sem{E}(\tau,
    \phi)}{\phi \in \Sem{\TracePat}(\tau) }.  \]

Our language constraints the structure of possible patterns.
As shown in Figure~\ref{fig:semantics}, a pattern consists in a
sequence of clauses, which can take one of three different forms:
$(t:\TransPat)$, $(\FirstAfter{t:\TransPat}{t'})$ and
$(\LastBefore{t:\TransPat}{t')}$. Here, $t$ and $t'$ are transition
variables and $T$ is a \emph{transition pattern}. In all three cases,
we say that $t$ is \emph{constrained} by the clause.

\subsection{Transition Patterns}\label{subsec:tpats-language}

A transition pattern can be thought as a predicate that takes as its
argument a pair $(\tau, \phi_{\Ag})$ of a transition and an agent
matching. Our current implementation supports specifying transition
patterns using KaSim's \emph{edit notation}. Transition patterns
defined this way are enclosed within curly brackets.  For example, in
query~(\ref{eq:exq1}) of section~\ref{sec:starting-example},
\[ \Set{ \AG{S}{x_{\Trans{u}{p}},\ d^{\,1}},\ \iAG{k}{K}{d^{\,1}} } \]
is true for a transition $t$ and a matching $\phi_{\Ag}$ if and only
if $t$ has the effect of phosphorylating a substrate that is bound to
the kinase with identifier $\phi_{\Ag}(k)$. Formally, a transition
pattern $T$ is interpreted as a function $\Sem{T}$ that maps
transitions into sets of agent matchings. Using the predicate
terminology, one may say that $\phi_{\Ag} \in \Sem{T}(t)$ if and only
if $(t, \phi_{\Ag})$ satisfies $T$.

Our query language can be instantiated with any choice of a language
specifying transition patterns. The only requirement is that
transition patterns should be {decidable efficiently} in the following
sense. Given a transition pattern $T$ and a transition $t$, one should
be able to efficiently compute whether $\Sem{T}(t)$ is empty and
generate an element of it in the case it is not. Our evaluation
algorithm relies on this property.

% TODO: rigidity condition, with clause

\subsection{Expression Language}\label{subsec:expr-language}

We show Figure~\ref{fig:expressions} the syntax of our expression
language. An expression can consist in an agent variable, a constant,
a parenthesized expression, a binary operation, a
function\footnote{Note that functions always take a single argument,
  which can be a tuple.} of an expression, a tuple of expressions or a
\emph{measure}.

Measures are the basic constructs through which information is
extracted from a trace. They come in two different kinds: \emph{state
  measures} and \emph{event measures}. State measures are used to
extract information about the state of the mixture at different points
in the trace. They are parametered with \emph{state expressions} that
can take the form $\StateBefore{t}$ or $\StateAfter{t}\,$, denoting
the states before and after transition $t$, respectively. For example,
the \texttt{int\_state} measure that is used in (\ref{eq:exq1}) is a
state measure. In addition, event measures are used to extract
information that is about a transition itself (in contrast to the
states that it connects). They are parametered by transition
variables. For example, the \texttt{time} measure that is used in
(\ref{eq:query-lifespan}) is an event measure.

The expression language can be easily extended with new operators,
functions, measures and types. In the same way than the language for
specifying transition patterns, it should be regarded as a parameter
of our query language and not as a rigid component.

% TODO: extensible, types, only ask for decidability

% -*- TeX-master: "main.tex" -*-

\begin{figure}[p]
\hrulefill
\centering
\begin{equation*}\label{tql-syntax}
  \arraycolsep=5pt
  \def\arraystretch{1.4}
  \begin{array}{rcclcl}
    \m{query} & Q & := & 
    \MatchRet{\TracePat}{E} \ \, 
    & \ &  \Sem{Q} \in \Trace \to \Bag{\Value} \\
    \m{pattern} & \TracePat & := & C
    & & \Sem{\TracePat} \in \Trace \to \Set{\Matching} \\
    &  & | &  C \,\textsf{and}\, C \\
    \m{clause} & C & := & t:\TransPat
     &  & \Sem{C} \in \Trace \to \Set{\Matching} \\
     &  & | & \FirstAfter{t:\TransPat}{t'} \\
     &  & | & \LastBefore{t:\TransPat}{t'} \\
    \m{transition pat.} &  \TransPat & := & \cdots  %\{ M \} \ \textsf{with} \ E
     & & \Sem{\TransPat} \in \Transition \to \Set{\Matching_{\Ag}}  \\
    \m{expression} & E & := & \cdots
     & & \Sem{E} \in \Trace \times \Matching \to \Value \\
  \end{array}
\end{equation*}
\smallskip
\begin{equation*}
    \def\arraystretch{1.9}
    \begin{array}{rcl}
     \Sem{\MatchRet{\TracePat}{E}}(\tau) & \ = \ &
     \BagC{\Sem{E}(\tau, \phi)}{\phi \in \Sem{\TracePat}(\tau)} \\
     % \Sem{C}(\tau) & = & \Sem{C}(\tau) \\
     \Sem{C \m{ and } C'}(\tau) & = & 
     \Sem{C}(\tau) \Inter \Sem{C'}(\tau) \\
     \Sem{ t : T }(\tau) & = & 
     \SetC{\phi}{\phi_{\Ag} \in \Sem{T}(\At{\tau}{\phi_{\Tr}(t)})} \\
  \end{array}
\end{equation*}
\smallskip
\[
 \Sem{ \FirstAfter{t:\TransPat}{t'} }(\tau)  \ = \  
     \LargeMultilineSetC{\phi}{
       \phi_{\Ag} \in \Sem{\TransPat}(\At{\tau}{\phi_{\Tr}(t)}) \,,\ 
         \, \phi_{\Tr}(t') < \phi_{\Tr}(t) \,, }{
       \forall i. \ \phi_{\Tr}(t') < i < \phi_{\Tr}(t) \implies
       \phi_{\Ag} \notin \Sem{\TransPat}(\At{\tau}{i})
     }
\]
\[
 \Sem{ \LastBefore{t:\TransPat}{t'} }(\tau)  \ = \
     \LargeMultilineSetC{\phi}{
       \phi_{\Ag} \in \Sem{\TransPat}(\At{\tau}{\phi_{\Tr}(t)}) \,,\ 
       \, \phi_{\Tr}(t) < \phi_{\Tr}(t') \,, }{
       \forall i. \ \phi_{\Tr}(t) < i < \phi_{\Tr}(t') \implies
       \phi_{\Ag} \notin \Sem{\TransPat}(\At{\tau}{i})
     }
\]

\medskip
\hrulefill
\smallskip

\caption{Syntax and semantics of the Trace Query Language}
\label{fig:semantics}
\end{figure} \newcommand{\GEvMeasure}{M_{\textsf{e}}}
\newcommand{\GStMeasure}{M_{\textsf{s}}}


\begin{figure}[p]
  \vspace{0.5cm}
  \hrulefill
  \centering  
  \begin{equation*}\label{expr-syntax}
  \arraycolsep=5pt
  \def\arraystretch{1.4}
  \begin{array}{rccl}
    \m{expression} & E & := &
        a \OR C \OR (\,E\,) \OR E \bowtie E \OR f\,(\,E\,) \OR E\,,\, E \OR \\
     & &  &  
        \GStMeasure[\,S\,] \OR \GStMeasure[\,S\,](\,E\,) \OR
        \GEvMeasure[\,e\,] \OR \GEvMeasure[\,e\,](\,E\,)  \\
    \m{constant} & C & := & 0 \OR 1 \OR \cdots 
          \OR \STR{foo} \,\OR \cdots \\
    \m{binary operator} & \bowtie & := & + \OR - \OR = \OR < \OR \cdots \\
     \m{function} & f & := & 
        \m{agent\_id} \OR \textsf{size} \OR \textsf{count} \OR \cdots \\
    \m{state measure} & \GStMeasure & := & \m{int\_state} \OR
         \m{component} \OR \m{snapshot} \OR \cdots \\
    \m{state expression} & S & := & \StateBefore{e} \OR \StateAfter{e} \\
    \m{event measure} & \GEvMeasure & := & \m{time} \OR
         \m{rule} \OR \cdots \\
    %\m{agent identifier} & a \\
    %\m{event identifier} & e \\
  \end{array}
  \end{equation*}

  {(with $a$ an agent variable and $e$ a transition variable)}

  \medskip
  \hrulefill
  \smallskip

  \caption{Syntax of expressions}
  \label{fig:expressions}
\end{figure}

\section{Evaluating Queries}

In this section, we introduce a natural subset of the language
described in section~\ref{sec:semantics}, for which we provide an
efficient implementation. Queries in this subset are said to be
\emph{regular}, and they display an interesting {rigidity} property.

\subsection{Rigidity}

Intuitively, a pattern is said to be rigid if its matchings are
completely determined by the value of a single transition variable.
\begin{definition}\label{def:rigidity}
  Given a Kappa model, a pattern $P$ is said to be \emph{rigid} if and
  only if it features a transition variable $r$ called \emph{root
    variable} such that for any trace $\tau$ that is valid in the
  model, we have
  \[ \forall\, \phi, \phi' \in \Sem{P}(\tau), \ \phi_{\Tr}\,(r) =
    \phi'_{\Tr}\,(r) \implies \phi = \phi'. \]
\end{definition}
For example, the pattern $P$ of query~(\ref{eq:query-lifespan})
% in section~\ref{sec:starting-example}
is rigid, with root variable $b$. Indeed, suppose that $b$ is matched
to a specific transition $t$. Then, the agent variable $s$ is
determined by $t$ as no more than one substrate can get bound during a
single transition given the rules of our model
(Figure~\ref{fig:model}). Finally, $u$ is uniquely determined as the
first unbinding event that targets $s$ after $b$.

An easy consequence of Definition~\ref{def:rigidity} is that the
number of matchings of a rigid pattern into a trace is bounded by the
size of this trace.

\subsection{Regular Queries}

Our evaluation algorithm handles a subset of queries whose patterns
admit a certain tree structure. For those patterns, rigidity is
implied by a weaker notion of \emph{local rigidity}.
\begin{definition}
  Given a Kappa model, a transition pattern $T$ is said to be
  \emph{rigid} if and only if for any agent variable $a$ that appears
  in $T$ and every valid transition $t$, we have
  \[ \forall\, \phi_{\Ag}\,, \phi_{\Ag}' \in \Sem{T}, \ \phi_{\Ag}(a)
    = \phi'_{\Ag}(a). \]
\end{definition}
Intuitively, a transition pattern is rigid if matching it to a
transition determines all its agent variables.
\begin{definition}
  Given a model, a pattern $P$ is said to be \emph{locally rigid} if
  it features only rigid transition patterns. Then, a transition
  variable $t$ is said to \emph{determine} an agent variable $a$ if
  there is a clause of $P$ that constrains\footnote{As defined in
    section~\ref{subsec:structure}.} $t$ and features $a$.
\end{definition}
For patterns with a particular structure, local rigidity implies
rigidity. This structural assumption can be expressed in terms of a
pattern's \emph{dependency graph}.
\begin{definition}
  The \emph{dependency graph} of a pattern $P$ is a graph whose nodes
  are the transition variables of $P$ and for which there is an edge
  from $t$ to $t'$ if and only if $P$ contains a clause of the form
  % \[\FirstAfter{t':T}{t} \qquad \text{or} \qquad
  %   \LastBefore{t':T}{t}.\]
  $(\FirstAfter{t':T}{t})$ or $(\LastBefore{t':T}{t})$.
\end{definition}
We can now define the notion of a regular pattern, and thus of a
regular query.
\begin{definition}\label{def:regularity}
  A pattern is said to be \emph{regular} if the following three
  conditions hold:
  \begin{inparaenum}[(i)]
  \item\label{reg:locally-rigid} it is locally rigid
  \item\label{reg:tree} its dependency graph is a tree
  \item\label{reg:well-captured} whenever two of its transition
    variables determine a same agent variable, one of them has to be a
    descendent of the other in the dependency tree.
  \end{inparaenum}
\end{definition}
This structure enables an efficient enumeration of the matchings of a
regular pattern into a trace. Moreover, the number of
these matchings is bounded by the size of the trace, as regular
patterns can be proven to be rigid.
\begin{proposition}\label{prop:regular-rigid}
  Regular patterns are rigid.
\end{proposition}
Finally, regular queries are defined as expected.
\begin{definition}
  A query is said to be \emph{regular} if its pattern is regular.
\end{definition}
This notion of regularity may appear unintuitive at first, and we
agree that its formal definition is somewhat involved. In practice
though, they should be recognized instinctively by experimentalists

% More precisely, the root of the dependency tree of a regular pattern
% is a root variable, in the sense of
% Definition~\ref{def:rigidity}. The intuition underlying this
% proposition should become clearer in section~\ref{subsec:evalq},
% when we introduce an algorithm for evaluating regular queries
% (defined below).

\subsection{Evaluating Regular Queries
  Efficiently}\label{subsec:evalq}

When designing an algorithm for evaluating queries, one has to keep in
mind that the corresponding sequence of state mixtures cannot fit in
random-access memory all at once, even for small traces. In fact, even
the most economic representation of a trace, which is specified by an
initial mixture and a sequence of labeled rewriting events, may fail
to fit in memory in some cases. Therefore, as often as possible, one
should only be allowed to stream such a representation from disk,
recomputing intermediate states dynamically and never keeping more
than a small number of them in memory at once (two in our case).

Our algorithm for evaluating a regular query proceeds in two
steps. First, it streams the trace to compute the set of all matchings
of the pattern. Then, it streams the trace a second time to compute
the value of the expression for all these matchings. The second step
is quite simple to implement. Indeed, once the matchings are known, it
is easy to compute the sequence of all measures that need to be
performed and order them in increasing order of time. The first step
attempts to match the root variable of the pattern to every transition
in the trace. For each candidate matching, it uses rigidity to
determine all other variables progressively as the trace is streamed,
in an order that is determined by the dependency tree and with a
minimal amount of caching. Overall, the algorithm runs in time linear
in the length of the trace.


\subsection{Our Implementation}

We provide an implementation of the trace query language, which relies
on the algorithm that is mentioned in section~\ref{subsec:evalq} for
evaluating regular queries. Our query engine takes for inputs
\begin{inparaenum}[(i)]
\item a file that contains a list of queries written in the same
  syntax that we use in our examples and
\item a trace file that has been generated by the Kappa simulator
  using the \texttt{-trace} option.
\end{inparaenum}
It evaluates all queries at once and generates one output file per
query, in comma-separated values (CSV) format.\footnote{Every line of
  an output file represents a single value. In our expression
  language, values are tuples of \emph{base values}. These are
  separated by commas within a line.}

Queries that are non-regular for structural reasons -- i.e. that fail to
meet criterion (\ref{reg:tree}) or (\ref{reg:well-captured}) of
Definition~\ref{def:regularity} -- are rejected immediately.  As there
is no easy static check for local rigidity, 
%-- criterion (\ref{reg:locally-rigid}) -- 
queries that do not meet this criterion may be rejected at runtime. 
% In both cases, informative error messages are generated.




\iffalse
However, the first users of our query engine never expressed any
frustration with it, as they naturally came up with regular queries
only.\footnote{Furthermore, they would also write the clauses of their
  patterns in an order that reflects their dependency trees.}  In our
opinion, this is due to the difficulty of interpreting non-regular
queries operationally.
\fi


%We provide an implementation

% How non regular are rejected ?
% CSV

\newpage 

\section{A Use Case on Wnt Signaling}


\subsection{Motivation and Problem Statement}

In this use case, we are focusing on a simplified model of the
$\beta$-catenin destruction complex from canonical Wnt signaling. This
complex is highly conserved in animals, and operates from humans to
nematodes to insects to amphibians, regulating the establishment of
the dorso-ventral axis. It is also heavily involved in colon cancer.

A source of complexity in our model is the fact that none of these
enzymes bind directly their substrate. Instead they are loaded onto a
scaffold. Moreover, the scaffold can head-to-tail homopolymerize, in
addition to having three independent binding sites on a second
scaffold, itself capable of dimerization. This allows a highly
degenerate complex of scaffolds, where connection paths or
stochiometries are dynamic. It is this complex that acts as a
super-scaffold to bring the substrate in contact with the
enzymes. Considering both scaffolds contain large regions of disorder
(i.e. chunks of unfolded peptide with high flexibility), it is
sensible to believe an enzyme loaded on one scaffold could modify the
substrate loaded on the neighboring scaffold. Lacking experimental
evidence to suggest a ballpark limit for this reachable horizon, we
leave it constrained: an enzyme will be able to modify any substrate
loaded onto its complex.

%TODO: remove future tense
Having kinases (i.e. enzymes that add a phosphate group) and
phosphatases (i.e. enzymes that remove a phosphate group) loaded on
the same complex will result in unimolecular do-undo loops. Unlike
the classic Goldbeter–-Koshland loop, the sensitivity of this kind of
system will not be simply on the abundance of forward/reverse enzymes,
but now we must consider how they are recruited to complexes, the size
of the complex they get loaded on, and the precedence relations
between the modifications.


\subsection{Experimental Protocol and Queries}

To explore this system, we create a Kappa model with three
parametrizations. The model contains the scaffold proteins Axin1 (Axn)
and APC, the kinases CK1$\alpha$ (CK1) and GSK3$\beta$ (GSK), the
protein phosphatases PP1 and PP2, and the substrate of all these
reactions, $\beta$-catenin (Cat). The destruction complex recruits
Cat through Axn. It then gets phosphorylated at the Serine on position
45 (S45) by CK1. While S45-phosphorylated, it can be phosphorylated at
the Threonine on position 41 (T41) by GSK. While T41-phosphorylated,
it can be phosphorylated on both Serines on positions 37 and 33 (S37 and
S33). Once S37 and S33-phosphorylated, Cat is degraded. Meanwhile, PP1
undoes the phosphorylations of CK1, while PP2 undoes those of GSK.

The three parametrizations are meant to explore the relationship
between phosphatase/kinase ratio and the distribution of do/undo
events per agent. The three parameter pairs are 50/10, 10/10, and
10/50, all in units of number of agents, for the number of kinases and
phosphatases in the model (e.g. 10/50 presents 10 copies of PP1, 10
copies of PP2, 50 copies of CK1, and 50 copies of GSK). The scaffolds
remain at an abundance of 100 each. The models begin with an initial
amount of Cat of 500 agents, and the models are simulated for 500
simulated seconds. We use a universal stochastic bi-molecular binding
rate of $10^{-4}$ per second per agent, a uni-molecular binding rate
of $10^{-2}$ per second, an unbinding rate of $10^{-2}$ per second,
and a catalytic rate of $1.0$ per second. 

For all three parametrizations, we simulate our model and run the
following queries on the resulting traces.

%%%%%%%%%%%%%%%%%%%%%%%%%%%%%%%%%%%%%%%%%%%%%%%%%%%%%%%%%%%%%%%%%%%%%%%%%%%%%%%%

\subsubsection*{Undoing S45, T41, S37 and S33 phosphorylation}
Considering phosphatases undoing the phosphorylation of
sites, does this happen to all agents? Does it happen to just a few agents? What is the distribution of dephosphorylation events per agent? (Figure~\ref{F1})

%match e:{c:Cat(S45{ph/un})}
%return (agent_id{c}, time[e])

%match e:{c:Cat(T41{ph/un})}
%return (agent_id{c}, time[e])

%match e:{c:Cat(S37{ph/un})}
%return (agent_id{c}, time[e])

%match e:{c:Cat(S33{ph/un})}
%return (agent_id{c}, time[e])

\newcommand{\UndoQ}[1]{
\Query{
    \m{match} & e:\Set{ 
      \iAG{c}{\BetaCat}{{#1}^{\,1}_{\Trans{p}{u}}}
    } \\
    \m{return} & \AgentId{c},\, \Time{e}
  }
}

\begin{small}
  \begin{equation*}
    \arraycolsep=7pt
    \begin{array}{cc}
      \UndoQ{S45} & \UndoQ{T41} \\ \\
      \UndoQ{S37} & \UndoQ{S33} \\
    \end{array}
  \end{equation*}
\end{small}

%%%%%%%%%%%%%%%%%%%%%%%%%%%%%%%%%%%%%%%%%%%%%%%%%%%%%%%%%%%%%%%%%%%%%%%%%%%%%%%%

\subsubsection*{Wait times}

What is the distribution of times spent
between the first phosphorylation on an agent, and the time it gets
degraded? (Figure~\ref{F2})

%match i:{+c:Cat}
%and first p:{c:Cat(S45{un/ph})} after i
%and first d:{-c:Cat} after p
%return (agent_id{c}, time[p], time[d])

\begin{small}
\begin{equation}
  \Query{
    \m{match} & i:\Set{  c:\BetaCat+ } \\
    \m{and} & \FirstAfter{p:\Set{
        \iAG{c}{\BetaCat}{{S45}_{\Trans{u}{p}}}
    }}{i} \\
    \m{and} & \FirstAfter{ d:\Set{
        c:\BetaCat-
    } }{p} \\
    \m{return} & \Time{d} - \Time{p}
  } 
\end{equation}
\end{small}

\noindent \textit{About this query.} Agent creation and destruction is
expressed by suffixing agent names with $+$ and $-$,
respectively.

%%%%%%%%%%%%%%%%%%%%%%%%%%%%%%%%%%%%%%%%%%%%%%%%%%%%%%%%%%%%%%%%%%%%%%%%%%%%%%%%

\subsubsection*{Component size and enzyme identity} 
Where do the phosphorylation steps that actually lead to degradation
occur? Do they happen mostly on large complexes? What is
the composition in units of Axn and APC of the complexes where the
phosphorylation events leading to degradation took place? What is the distribution of kinase identifiers for the last phosphorylation events that lead to degradation? (Figure~\ref{F6})

\newcommand{\BigHectorStoryLine}[4]{
\LastBefore{#1:\Set{ 
          \iAG{c}{\BetaCat}{ {#2}^{\,1}_{\Trans{u}{p}}}, \ 
          \iAG{#3}{#4}{c^{\,1}}
    }}{d}
}
\newcommand{\BigHectorStoryRet}[2]{
\AgentId{#2}, \ \Count{ \Component{\StateBefore{#1}}{#2}, \, 
      \STR{Axn}, \, \STR{APC} }
}

%match d:{-c:Cat}
%and last p1:{ c:Cat(S45{un/ph}[1]), k1:CK1(c[1])} before d
%and last p2:{ c:Cat(T41{un/ph}[1]), k2:GSK(c[1])} before d
%and last p3:{ c:Cat(S37{un/ph}[1]), k3:GSK(c[1])} before d
%and last p4:{ c:Cat(S33{un/ph}[1]), k4:GSK(c[1])} before d
%return (
%	agent_id{k1}, count{'Axn', 'APC'}{component[.p1]{k1}},
%	agent_id{k2}, count{'Axn', 'APC'}{component[.p2]{k2}},
%	agent_id{k3}, count{'Axn', 'APC'}{component[.p3]{k3}},
%	agent_id{k4}, count{'Axn', 'APC'}{component[.p4]{k4}}

\begin{small}
\begin{equation}\label{query:cat-deg}
  \Query{
    \m{match} & d:\Set{ c:\BetaCat - } \\
    \m{and} & \BigHectorStoryLine{p_1}{S45}{k_1}{\CKOne} \\
    \m{and} & \BigHectorStoryLine{p_2}{T41}{k_2}{\GSK}   \\
    \m{and} & \BigHectorStoryLine{p_3}{S37}{k_3}{\GSK}   \\
    \m{and} & \BigHectorStoryLine{p_4}{S33}{k_4}{\GSK}   \\
    \m{return} 
    & \BigHectorStoryRet{p_1}{k_1} \,,\  \\
    & \BigHectorStoryRet{p_2}{k_2} \,,\ \\
    & \BigHectorStoryRet{p_3}{k_3} \,,\  \\
    & \BigHectorStoryRet{p_4}{k_4} \\
  }
\end{equation}
\end{small}

\noindent \textit{About this query.} The \textsf{component} state
measure computes the connected component that contains an agent in a
mixture. It returns a set of agents $S$. The \textsf{count} function
takes such a set $S$ along with $n$ strings denoting agent types and
returns an $n$-tuple of integers indicating how many agents of each
type appear in $S$.


\subsection{Results and Interpretation}

\subsubsection*{Concentration Time Traces}
From the simulator’s output, we get the evolution of the abundance of
Cat through time. In Figure~\ref{F0}, we can see the systems with low phosphatase
behave similarly, even though one has five times the amount of kinases
than the other (blue vs red traces). In contrast, the system with high
phosphatase shows markedly less degradation of Cat; where the other
two system degraded around 450 units, this one has only degraded
23. From this whole-system view, it would seem the amount of
phosphatase is more critical than the amount of kinase: based on the
1:1 system, increasing the kinase five fold has little effect, whereas
increasing the phosphatase has a more dramatic effect.



\begin{figure}[p]
  \centering
  \includegraphics[scale=0.3]{wnt/F0_abundance_of_cat_through_time}
  \caption{Tracking the abundance of agent Cat through the
    simulation. At time $T=0$, the agents are introduced, all in
    monomeric form. The simulation was stopped after five hundred
    simulated seconds. In this legend and throughout the figures, ph
    stands for phosphatase, ki stands for kinase, and the agent
    numbers are presented. Thus ``$10$ ph : $50$ ki'' means the system
    with $10$ units of phosphatase and $50$ of kinase.}
  \label{F0}
\end{figure}

\subsubsection*{Distribution of undo events per agent}

To further study the effect of adding phosphatase, we now look at the
distribution of dephosphorylation events per agent, for each of the
four residues, in Figure~\ref{F1}. S45 is the first residue to
phosphorylate in the causal chain leading to degradation. Based on the
1:1 system, is is surprising to see increasing the phosphatase level
five fold maintains a similar total number of dephosphorylation events
(compare curves’ integrals). However, their distribution is quite
different; under 1:1 conditions the dephosphorylation events happen
mostly on a sub-population of agents, while under 5:1 conditions they
are happening more widespread. Interestingly, the increasing the
amount of kinase to 1:5 led to decrease in dephosphorylation
events. It is also worth noting, the 1:1 saw almost 30 thousand
dephosphorylation events of S45, with some specific agents getting
dephosphorylated almost 800 times. This would imply a comparable
number of phosphorylation events for that specific copy of Cat.

To answer the question that motivated this query, under 1:5 regime,
most agents don’t get sabotaged by the phosphatase; the blue line is
quite flat. Decreasing the amount of kinase however changes this, and
under a 1:1 regime some agents get undone several times, a quarter
seeing upwards of hundreds of undo events (e.g. from id 300
onward). Increasing the phosphatase to a 5:1 regime further
exacerbates this, with over half the agents receiving undo events
hundreds of times. The unavailability of phosphorylated S45 in turn
inhibits the phosphorylation of T41, and so forth to S37 and S33. It
is worth noting that, based on the 1:1 system, increasing the
phosphatase five fold decreases the number and extend of advanced
dephosphorylation events, such as S33 and S37. Paradoxically,
increasing the kinase five fold has this same effect. We attribute the
former to decreased availability of the intermediate phosphorylated
states (i.e. if T41 is not phosphorylated, S37 can’t be
phosphorylated, ergo can’t be dephosphorylated), and the latter to
increased throughput to degradation (i.e. Cat is not around for long
enough to get dephosphorylated, as once it gets phosphorylated it
extremely quickly proceeds to get degraded).

We call attention to the number of agents whose final sites got
dephosphorylated (Figure~\ref{F1}), vs the number of agents who got
degraded (Figure~\ref{F0}). The 1:5 or 1:1 systems both degraded over
450 agents each, but the former undid around 160 agents while the
latter undid over 350. For the 1:5 and 5:1 systems, both undid around
160 agents, but the former degraded over 450 agents while the latter
less than 50. This argues the notion of efficiency (e.g. minimizing
the amount of undo steps) can’t readily be inferred from the
throughput of the system.


\begin{figure}[p]
  \centering
  \includegraphics[scale=0.35]{wnt/F1_distribution_dephosphorylations_per_agent}
  \caption{Distribution of dephosphorylation events per agent. Each
    time an agent gets dephosphorylated, its ID is registered. After
    sorting, we plot the distribution of these IDs for all four
    residues in the three parameter regimes. The area under the curve
    is also presented on each legend.}
  \label{F1}
\end{figure}


\subsubsection*{Wait times}
Looking at the distribution of wait times (Figure~\ref{F2}), from first phosphorylation
to degradation, we note the bulk of degradation events occur rapidly,
in less than $50$ seconds. Worth noting that, from the 1:1 regime,
increasing the amount of kinase five fold marginally reduced wait
times. Increasing the amount of phosphatase produced such sparse
results that little can be said.

\begin{figure}[p]
  \centering
  \includegraphics[scale=0.35]{wnt/F2_wait_times}
  \caption{Distribution of wait times from first phosphorylation until
    degradation. The sum of the bins is presented in the title of each
    plot. The number of bins was determined by MatLab's automatic
    algorithm. A “forward model execution” means a degradation event.}
  \label{F2}
\end{figure}

\subsubsection*{Distribution of dephosphorylation times}

We wondered if the dephosphorylation events could be due to transient
recruitment of phosphatases to complexes, if the system operated
mostly in the forwards (phosphorylation) route, with occasional bursts
of dephosphorylation activity. We therefore plotted the distribution
of dephosphorylation events in time (Figure~\ref{F3}).

Based on the system with 1:1, increasing the amount of kinase five
fold to 1:5 flattened the distribution of dephosphorylation times. In
contrast, increasing the amount of phosphatase had the opposite
effect, with the bulk of events happening after half the simulation
had elapsed. For the 1:1 system, we interpret the early stages as
revealing an initialization wait: scaffold complexes are being
assembled, and so the phosphatase events are not happening as
fast. Once these are assembled, the system enter a more stable
regime. However, eventually the abundance of Cat is low enough that
the system becomes less active. These regimes match the different
regions in Figure~\ref{F0}, with a slow decline before time $T=100$, a
steady decline through to $T=400$, then a slow decline onward.


\begin{figure}[p]
  \centering
  \includegraphics[scale=0.35]{wnt/F3_distribution_of_dephosphorylations_in_time}
  \caption{Distribution of dephosphorylation events in time.}
  \label{F3}
\end{figure}

\subsubsection*{Responsibility of final phosphorylation}

We next turned to the question of how did enzymes contribute to the
forward kinetics, those of phosphorylation. In Figure~\ref{F4}, we
show how many times a particular enzyme produced the final
phosphorylation before degradation. In Figure~\ref{F5}, we show the
same but as a fraction of all degradation events. Under both 1:5 and
1:1 regimes, the distributions are fairly flat. In contrast, under the
5:1 regime, it is only a few enzymes that contributed to the actual
degradation of Cat; all the other thousands of phosphorylation and
dephosphorylation events (Figure~\ref{F1}) were caught in the do-undo
cycle.

\begin{figure}[p]
  \centering
  \includegraphics[scale=0.35]{wnt/F4_final_phosphorylation_enzyme_id_absolute}
  \caption{Identifiers of enzymes responsible for the final
    phosphorylation step taken before degradation, for all four
    residues. Presented are absolute counts per enzyme id.}
  \label{F4}
\end{figure}


\begin{figure}[p]
  \centering
  \includegraphics[scale=0.35]{wnt/F5_final_phosphorylation_enzyme_id_fraction}
  \caption{Identifiers of enzymes responsible for the final
    phosphorylation step taken before degradation, for all four
    residues. Presented are fractions of overall contribution per
    enzyme id. $\bar{x}$ represents the average number of final
    substrate modification per enzyme, $\sigma$ the standard
    deviation.}
  \label{F5}
\end{figure}


\subsubsection*{Complex composition: changes in ratio}

Another way of looking at the question of what complex contributes
most to the degradation of Cat is to query the size of the complex at
the last phosphorylation event before degradation. Taking S45 as
representative of all the other residues, we plot the size of the
complex, in terms of Axn and APC. Overall, we see a broad distribution
of sizes, with a some phosphorylation events occurring in large
complexes (i.e. $>80$ Axn, $>40$ APC), but a significant number occurring
in far smaller complexes (i.e. $<10$ Axn, $<10$ APC). Changing the
parameter regime of kinase to phosphatase does not seem to alter this
behavior significantly, though for the 1:5 system the number of events
is low enough that it is difficult contrasting against the behavior of
the other systems. As for the other residues, their behavior is
extremely similar to the one shown for S45.

\begin{figure}[p]
  \centering
  \includegraphics[scale=0.35]{wnt/F6_complex_composition_final_S45}
  \caption{Composition of the complex, in terms of Axn and APC
    components, at the last event where Cat got S45 phosphorylated
    before being degraded. The number of points is the number of
    degradation events, per system. This scatter plot’s points have
    been nudged with a random noise factor of 0.2 to increase visual
    perception of discrete points where the data overlap.}
  \label{F6}
\end{figure}


\subsubsection{Complex composition: all four on the same component?}

For the final query, we wonder if all four final phosphorylation
events occur on the same complex. Given the short wait time
(Figure~\ref{F2}), one might expect so, but the number of
dephosphorylation events is so large (Figure~\ref{F1}), it could be
well that a substrate is partially modified on one complex,
subsequently modified on another, finalized in yet another. Lacking a
metric by which we can compare complexes for distance, we instead
compare complex compositions as a proxy.

Seeing how overwhelmingly, for each specific Cat’s modifications, the
S45 phosphorylation events happened on complexes of the same Axn and
APC composition as the T41 phosphorylation events, as the S37
phosphorylation events, as the S33 phosphorylation events, we feel
confident in claiming all four steps occurred on the same
complex. This agrees well with the observation that the wait times are
fairly short (Figure~\ref{F3}). We did not see an appreciable
difference for the other parameter regimes.

\begin{figure}[p]
  \centering
  \includegraphics[scale=0.32]{wnt/F10_complex_composition_10_50}
  \caption{Complex composition at the time of the last phosphorylation
    for the 1:5 system. All four residues are shown. A diamond
    superposed with a cross superposed with a circle superposed with a
    plus sign indicates all four modifications for a specific copy of
    Cat occurred on a complex of the same composition in terms of Axn
    and APC. We interpret this as having occurred on the exact same
    complex.}
  \label{F10}
\end{figure}

\subsection{Conclusions}

\begin{enumerate}
\item The number of undo events does not inform us of overall
  throughput (Figure~\ref{F0} and Figure~\ref{F1}).
\item How a step may be affected by changing abundances depends
  greatly on its upstream context (Figure~\ref{F1}).
\item Entities that got degraded waited a short while since their
  first modification (Figure~\ref{F2}), and yet most modifications
  were futile (Figure~\ref{F1}).
\item How the kinetics of dephosphorylation are distributed through
  time changes on the ratio of forward to reverse enzymes
  (Figure~\ref{F3}), with some regimes showing behavior absent in
  others.
\item We can’t argue that a specific enzyme performs the bulk of the
  effective (i.e. final) phosphorylation events (Figures~\ref{F4} and
  \ref{F5}), unless the reverse enzyme is too high, point at which a
  few enzymes contribute the bulk of the events.
\item We can’t argue that giant complexes, nor small complexes, nor
  medium complexes, are the sole entities responsible for performing
  the effective (i.e. final) phosphorylation events
  (Figure~\ref{F6}); %F6-9
  the distribution of complexes is wide, and they all contribute to
  the kinetics.
\item Once an agent starts getting modified, it keeps getting modified
  and finishes getting modified within the same complex
  (Figure~\ref{F10}); % (F10-12)
  a specific agent’s modification is not happening piece-meal over a
  large set of complexes, but rather quickly (Figure~\ref{F2}), on the
  same complex (Figure~\ref{F10}), % (F10-12)
  by a large set of enzymes (Figures~\ref{F4} and \ref{F5}).
\end{enumerate}

% Say a few words about the speed of the implementation

\newpage

\section{Conclusion}

% Interaction with causal analysis Insist on the clean semantics
% foundations Natural queries are natural

% \input{playground.tex}

% Extensions and Future work

%%%%%%%%%%%%%%%%%%%%%%%%%%%%%%%%%%%%%%%%%%%%%%%%%%%%%%%%%%%%%%%%%%%%%%%%%%%
% ---- Bibliography ----

\nocite{*} \bibliographystyle{splncs04} \bibliography{main}

\newpage

\appendix

%\section{Matchings Enumeration Algorithm}\label{ap:algo}


Let $Q$ a regular query with pattern $P$ and $\tau$ a trace. We first
discuss an algorithm to compute the set of matchings $\Sem{P}(\tau)$
of $P$ into $\tau$ in the special case where $P$ contains only clauses
of the form $(t:T)$ and $(\FirstAfter{t:T}{t'})$. Also, in order to
simplify our argument, we make the additional assumption that each
transition variable is constrained by exactly one clause of the form
($t:T$). This assumption can be easily removed by introducing a notion
of conjunction for transition patterns, along with a \emph{unit}
transition pattern that matches anything. We call $T$ the
\emph{primary constraint} on $t$. In addition, by
Definition~\ref{def:regularity}, every transition variable $t$ with the
exception of the root of $P$'s transition graph is constrained by a
unique clause of the form $(\FirstAfter{t:T}{t'})$. Here, we call $T$ the
\emph{positioning constraint} of $t$.

\bigskip

% \newcommand{\PMS}{\textsc{pms}}
\newcommand{\PMS}{p.m.s}

The state of our algorithm features a collection of \emph{partial
  matching structures}, that are defined as follows.
\begin{definition}
  In the context of a regular pattern and of a trace $\tau$, a
  \emph{partial matching structure} (abbreviated \PMS) consists in the
  following elements:
  \begin{inparaenum}[(1)]
  \item a \emph{partial transition matching} $p_{\!\Tr}$ that maps a
    subset of transition variables to transitions of $\tau$
  \item a \emph{partial agent matching} $p_{\!\Ag}$ that maps a
    subset of agent variables to agent identifiers
  \item a forest $F$ that consists in disjoint subtrees of the
    pattern's dependency graph.
  \end{inparaenum}

  In addition, the following invariants must hold:
  \begin{inparaenum}[(1)]
  \item trees in $F$ along with the domain of $p_{\!\Tr}$ form a
    partition of the set of all transition variables in the pattern
  \item if $t$ belongs to the domain of $p_{\!\Tr}$ and determines an
    agent variable $a$, then $a$ belongs to the domain of $l$.
  \end{inparaenum}
\end{definition}

% TODO: conjunction

Let $r$ the root of $P$'s dependency graph. Our algorithm starts with
an empty collection of \PMS. It streams the trace in order and does
the following for each transition $t$.

% TODO: Critical thing: what does it mean for a transition pattern to
% match a transition.

%\begin{enumerate}
%\item If the primary constraint of $r$ matches $t$, add
%\end{enumerate}

\end{document}