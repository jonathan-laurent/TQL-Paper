\section{Matchings Enumeration Algorithm}\label{ap:algo}


Let $Q$ a regular query with pattern $P$ and $\tau$ a trace. We first
discuss an algorithm to compute the set of matchings $\Sem{P}(\tau)$
of $P$ into $\tau$ in the special case where $P$ contains only clauses
of the form $(t:T)$ and $(\FirstAfter{t:T}{t'})$. Also, in order to
simplify our argument, we make the additional assumption that each
transition variable is constrained by exactly one clause of the form
($t:T$). This assumption can be easily removed by introducing a notion
of conjunction for transition patterns, along with a \emph{unit}
transition pattern that matches anything. We call $T$ the
\emph{primary constraint} on $t$. In addition, by
Definition~\ref{def:regularity}, every transition variable $t$ with the
exception of the root of $P$'s transition graph is constrained by a
unique clause of the form $(\FirstAfter{t:T}{t'})$. Here, we call $T$ the
\emph{positioning constraint} of $t$.

\bigskip

% \newcommand{\PMS}{\textsc{pms}}
\newcommand{\PMS}{p.m.s}

The state of our algorithm features a collection of \emph{partial
  matching structures}, that are defined as follows.
\begin{definition}
  In the context of a regular pattern and of a trace $\tau$, a
  \emph{partial matching structure} (abbreviated \PMS) consists in the
  following elements:
  \begin{inparaenum}[(1)]
  \item a \emph{partial transition matching} $p_{\!\Tr}$ that maps a
    subset of transition variables to transitions of $\tau$
  \item a \emph{partial agent matching} $p_{\!\Ag}$ that maps a
    subset of agent variables to agent identifiers
  \item a forest $F$ that consists in disjoint subtrees of the
    pattern's dependency graph.
  \end{inparaenum}

  In addition, the following invariants must hold:
  \begin{inparaenum}[(1)]
  \item trees in $F$ along with the domain of $p_{\!\Tr}$ form a
    partition of the set of all transition variables in the pattern
  \item if $t$ belongs to the domain of $p_{\!\Tr}$ and determines an
    agent variable $a$, then $a$ belongs to the domain of $l$.
  \end{inparaenum}
\end{definition}

% TODO: conjunction

Let $r$ the root of $P$'s dependency graph. Our algorithm starts with
an empty collection of \PMS. It streams the trace in order and does
the following for each transition $t$.

% TODO: Critical thing: what does it mean for a transition pattern to
% match a transition.

%\begin{enumerate}
%\item If the primary constraint of $r$ matches $t$, add
%\end{enumerate}