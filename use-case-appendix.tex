\section{Use Case Appendix}\label{ap:use-case}

\subsubsection*{Concentration Time Traces}
From the output of the simulator, we get the evolution of the
abundance of Cat through time. In Figure~\ref{F0}, we can see that the
systems with low phosphatase behave similarly, even though one has
five times the amount of kinases than the other (blue vs red
traces). In contrast, the system with high phosphatase shows markedly
less degradation of Cat; where the other two systems degraded around
450 units, this one has only degraded 23. From this whole-system view,
it would seem the amount of phosphatase is more critical than the
amount of kinase: based on the 1:1 system, increasing the kinase
five-fold has little effect, whereas increasing the phosphatase has a
more dramatic effect.

\begin{figure}[h]
  \begin{center}
    \includegraphics[width=0.78\columnwidth]{wnt/F0_abundance_of_cat_through_time}
  \end{center}
  \caption{Tracking the abundance of agent Cat through the
    simulation. At time $T=0$, the agents are introduced, all in
    monomeric form. The simulation was stopped after five hundred
    simulated seconds. In this legend and throughout the figures,
    ``ph'' stands for phosphatase, ``ki'' stands for kinase, and
    numbers indicate agent multiplicity. Thus ``$10$ ph : $50$ ki''
    means the system with $10$ units of phosphatase and $50$ of
    kinase.}
  \label{F0}
\end{figure}


\subsubsection{Complex composition: all four on the same component?}

For the final query, we wonder if all four final phosphorylation
events occur on the same complex. Given the short wait time
(Figure~\ref{F2}), one might expect so, but the number of
dephosphorylation events is so large (Figure~\ref{F1}), it could be
well that a substrate is partially modified on one complex,
subsequently modified on another, finalized in yet another. Lacking a
metric by which we can compare complexes for distance, we instead
compare complex compositions as a proxy.

Seeing how overwhelmingly, for each specific modification on a single
Cat, the S45 phosphorylation events happened on complexes of the same
Axn and APC composition as the T41 phosphorylation events, as the S37
phosphorylation events, as the S33 phosphorylation events, we feel
confident in claiming all four steps occurred on the same
complex. This agrees well with the observation that the wait times are
fairly short (Figure~\ref{F2}). We did not see an appreciable
difference for the other parameter regimes.

\begin{figure}[h]
  \centering
  \includegraphics[width=\columnwidth]{wnt/F11_complex_composition_10_10}
  \caption{Complex composition at the time of the last phosphorylation
    for the 1:1 system. All four residues are shown. A diamond
    superposed with a cross superposed with a circle superposed with a
    plus sign indicates that all four modifications for a specific
    copy of Cat occurred on a complex of the same composition in terms
    of Axn and APC. We interpret this as having occurred on the exact
    same complex.}
  \label{F11}
\end{figure}